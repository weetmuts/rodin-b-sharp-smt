%% ------------------------------------------------------- %% 
%% ------------------------------------------------------- %%

\documentclass[10pt,a4paper]{report}

\usepackage{enumerate}
\usepackage{bsymb}
\usepackage{url} 
\usepackage[english]{babel}
\usepackage[top=3cm,bottom=2cm,left=2cm,right=2cm]{geometry}
%\usepackage[colorlinks=true,urlcolor=black,linkcolor=black,citecolor=black]{hyperref}
\usepackage{hyperref}
\usepackage{array}
\usepackage{amsmath}
\usepackage{graphicx}

\graphicspath{{./images/}}

%% ------------------------------------------------------- %%
\title{ANR-08-DEFI-005 \\ DECERT \\ ~ \\ Task 7: Integrating decision procedures into Rodin \\ ~ \\ D14: Preliminary report on integrating SMT proof witnesses into Rodin}
\author{Systerel}

%% ------------------------------------------------------- %%
\begin{document}
%% ----------------------------- %%
\maketitle

%% ----------------------------- %%
%% Preambule 
%% ----------------------------- %%
\begin{abstract}

The Rodin platform \cite{RODIN} mainly supports two kind of activities: modeling with the event-B notation and discharging the sequents generated from the models. 
Some sequents are typical mathematical lemmas, which are often encountered in practice, but are not yet well supported by proving tools such as those embedded in the Rodin platform.
The objective of the task 7 is to integrate SMT solvers to the Rodin proving framework to enhance its proving capability, especially in the field of arithmetics.

\end{abstract}

\tableofcontents

%% ----------------------------- %%
%% Introduction %%
%% ----------------------------- %%
\section{Introduction}
This document will first introduce RODIN and the Event-B platform in order to show how integration can be made with SMT solvers. Then we will expose the translation mechanism from an event-B expression to an SMT-Lib 1.2 and an SMT-Lib 2.0 expression. Benchmarks have been performed to compare solvers efficiency according to a sequent type, and a solver analysis will be introduced in this document.  

\paragraph{}
This integration raises several issues:
\begin{itemize}
\item A translation scheme from the event-B mathematical notation (i.e. set-theory and arithmetics) to the formalism(s) understood by the decision procedures needs to be defined. Of course, not all
concepts of the event-B mathematical language will be translatable, as it has the same expressive
power as higher-order logic.
\item A filtering mechanism is needed, so that decision procedures are attempted (either automatically or interactively) only in cases where they could be useful. This mechanism will have strong links with the translation, as it would be useless to launch the reasoner on an untranslatable lemma.
\item The Rodin prover insists that reasoners return a minimal set of needed hypotheses, in order to foster proof reuse. Fortunately, this information can be extracted from the decision procedure
traces or certificates, as one purpose of such artefacts is to memorize this information.
\end{itemize}

%% ----------------------------- %%
%% Body
%% ----------------------------- %%
%% ------------------------------------------------------- %%
%% Rodin
%% ------------------------------------------------------- %%
\section{Event-B and the Rodin platform}
Event-B is a formal method for system modeling. Key features of Event-B are the use of set theory as a modeling notation, the use of refinement to represent systems at different abstraction levels and the use of mathematical proof to verify consistency between refinement levels.

The Rodin platform is an Eclipse\cite{ECLIPSE}-based IDE for Event-B that provides effective support for refinement and mathematical proof. The releases\cite{RODIN} of the Rodin platform as well as the source files\cite{SOURCES} are available from the SourceForge site. The tool documentation for users and developers is provided within the Event-B wiki\cite{WIKI}.

\subsection{Contributing to the Rodin Platform}
The Rodin platform is Open Source and is extendable with plug-ins and extension points.

\begin{figure}
\centering
\includegraphics[scale=0.25]{Rodin.png}
\caption{Plug-ins and extension points} 
\label{Fig:Rodin Platform}
\end{figure}

In particular, it is possible to add new reasoners, which will schematically run as follows:
\begin{itemize}
\item Input: a sequent Hypotheses $\vdash$ Goal.
\item Invocation of an SMT solver.
\item Output: a proof rule.
\end{itemize}

\subsection{Integration of a SMT Solver into Rodin}
\begin{itemize}
\item A formula $A$ is \textit{valid} in a theory $T$ if $A$ evaluates to $true$ in every model $M$ of $T$.
\item A formula $A$ is \textit{satisfiable} in a theory $T$ if there is a model $M$ for $T$ in which $A$ evaluates to $true$. Otherwise, $A$ is \textit{unsatisfiable}.
\end{itemize}

It can be deduced that if a formula is unsatisfiable, its negation is valid. In other words, if $A \land \lnot B$ is unsatisfiable, then $A \limp B$ is valid. As a consequence, an Event-B sequent is not passed as is to an SMT solver, but its negation is taken as input by the SMT solver.
Conversely, if a formula is satisfiable, it means that a counter-example exists for its negation.

\begin{figure}
\centering
\includegraphics[scale=0.5]{Integration_SMT.png}
\caption{Integration of an SMT solver} 
\label{Fig:SMT solver}
\end{figure}

The main difficulties are all related to the translation steps. They cover several sub-tasks, such as selecting the hypotheses, choosing the SMT solver to be addressed, determining the format of the benchmark to be generated (SMT-LIB 1.2 or SMT-LIB 2.0), keeping a link between Event-B hypotheses and the associated hypotheses in SMT-LIB format...\newline


SMT plugins can be found at the following address \url{https://rodin-b-sharp.svn.sourceforge.net/svnroot/rodin-b-sharp/trunk/_exploratory/fages/}:
\begin{itemize}
\item fr.systerel.smt.provers.core divided in following packages:
	\begin{itemize}
	\item br.ufrn.smt.solver.preferences (Preference page creation in Rodin to handle solvers' configuration)
	\item br.ufrn.smt.solver.translation (Event-B Proof Obligations to SMT translation)
	\item fr.systerel.smt.provers.ast (Description of SMT language tree structure)
	\item fr.systerel.smt.provers.ast.commands (Description of SMT commands tree structure)
	\item fr.systerel.smt.provers.ast.responses (Description of SMT responses tree structure)
	\item fr.systerel.smt.provers.core (Entry point to communicate with SMT solvers) 
	\item fr.systerel.smt.provers.internal.core (The plugin core)
	\end{itemize}
\item fr.systerel.smt.provers.ui 
	\begin{itemize}
	\item fr.systerel.smt.provers.internal.ui (Entry point to communicate with Rodin platform User Interface)
	\end{itemize}
\item fr.systerel.smt.provers.tests (Tests for Event-B to SMT translation)
\end{itemize}
%% ------------------------------------------------------- %%
%% SMT-LIB 1.2
%% ------------------------------------------------------- %%
\section{From Event-B to SMT-LIB 1.2}

\subsection{Titre}

\paragraph{}

\section{Targeted SMT Solvers}
\begin{itemize}
\item VeriT.
\item Alt-Ergo.
Tous deux sont des solveurs sur lesquels ont travaille des partenaires de DECERT.
Suite a la derniere reunion DECERT on peut esperer avoir de bons taux de reponses avec Alt-Ergo sur des problemes d'arithmetique lineaire.
\end{itemize}


Another solution is to represent sets by
their characteristic predicate, i.e. a boolean-valued function f taking as argument a value of the corresponding carrier
set. This solution can only deal with basic sets (no set of sets) but provides a
direct mapping to first-order logic with uninterpreted functions and predicates, for which efficient reasoning engines
are available.
%% ------------------------------------------------------- %%
%% SMT-LIB 2.0                                             
%% ------------------------------------------------------- %%
\section{SMT-LIB 2.0}
\subsection{Titre}
Presenter ici les nouveautes SMT 2.0
Considere l'interaction entre les solveurs SMT et d'autres outils.
Le chapitre V definit un langage de script (commandes, reponses) pour echanger des informations avec les solveurs SMT.

\begin{figure}
\centering
\includegraphics[scale=0.5]{SMT20.png}
\caption{SMT-LIB 2.0 Script Commands} 
\label{Fig:SMT-LIB 2.0}
\end{figure}

\subsection{Titre}
Interet.
\begin{itemize}
\item Information complementaire avec le livrable D7.
Le format des preuves retournees par get-proof n'est pas specifie dans SMT-LIB 2.0.
\item Information tres int�ressante pour l'integration dans Rodin.
\end{itemize}

\section{From Event-B to SMT-LIB 2.0}

\subsection{Titre}

\paragraph{}


\section{Targeted SMT solvers}
Voila un petit statut sur ce que j'ai pu observer sur les differents prouveurs SMT :
\begin{itemize}
\item CVC3. 
D'apres leur site, il supporte tous les formats SMT-LIB donc 2.0. 
Les logiques supportees (en tout cas celles presentees � la competition SMT 2010): $UFLRA$, $QF_UF$, $QF_RDL$, $QF_IDL$, $QF_BV$, $QF_UFIDL$, $QF_AX$, $AUFLIA+p$, $AUFLIA-p$, $AUFLIRA$, $QF_AUFLIA$, $QF_UFLRA$, $QF_UFLIA$, $QF_LRA$, $AUFNIRA$, $UFNIA$, $UFNIA+p$, $QF_LIA$, $QF_NIA$, $QF_UFNRA$, $QF_NRA$, $QF_ABV$.

\item CVC4. 
Prototype perdant beaucoup des fonctionnalites de CVC3 mais permettant d'avoir de meilleures perf sur la logique $QF_LRA$ de CVC3. Donc a mon avis pas tres interessant dans notre cas.

\item MathSAT 5. 
MathSat supporte a priori tout format SMT et possede des options permettant la generation de preuve et la recup�ration d'un  unsat core.
Logiques supportees (en tout cas celles presentees � la competition Smt 2010) : $QF_UF$, $QF_UFLRA$, $QF_UFLIA$, $QF_LRA$, $QF_LIA$.

\item MiniSmt. 
Supporte a priori tout format SMT.
Logiques supportees: $QF LIA$, $QF LRA$, $QF NIA$, $QF NR$. Celles presentees a la competition SMT 2010 sont les suivantes: $QF_NIA$, $QF_NRA$.

\item Open Smt. 
Arrive a parser des formats SMT-LIB 2.0 et possede options permettant la generation de preuve et recuperer un unsat core. Toutes les fonctionnalites SMT-LIB 2.0 ne sont pas encore supportees et une version prevue fin aout est censee etre plus stable.
Logiques supportees: $QF UF$, $QF IDL$, $QF RDL$, $QF LRA$, $QF UFIDL$ ( + $QF BV$, $QF AX$ et $QF UFLRA$ mais pas completement cf parser pas complet).

\item VeriT. 
Arrive a parser des formats SMT-LIB 2.0 mais ne supporte pas toutes les fonctionnalites SMT-LIB 2.0; il possede des options permettant la generation de preuve (pas de renseignements sur l'unsat core).
Logiques supportees (en tout cas celles presentees � la competition SMT 2010): $QF_UF$, $QF_RDL$, $QF_IDL$, $QF_UFIDL$.

\item Z3 (non presente � la competition). 
Supporte une partie de SMT-LIB 2.0 mais surement pas la globalite. Possede une option pour recuperer un unsat core et pour generer une preuve.
Attention, le get-unsat-core ne retourne pas necessairement les hypotheses telles qu'elles ont ete entrees. Celles-ci peuvent en effet avoir subi des transformations par le solveur.

\end{itemize}

\section{}
Preciser ici comment on remonte l'information de contre-exemple a l'utilisateur.

Preciser aussi comment on etablit un lien entre les hypotheses au format Event-B et les hypotheses au format SMT-LIB.
%% ------------------------------------------------------- %%
%% Rodin SMT plugin example
%% ------------------------------------------------------- %%
\section{Rodin SMT plugin example}
The RODIN plugin introduced in this document is still a prototype. For the moment, it can handle some basic arithmetic proof obligations but still needs to be improved to be much more efficient (TODO reference to Future Works). 
Here is an overview of how the plugin works:

\paragraph{Rodin preferences}
To use the plugin, the user must set up SMT preferences by reaching the Windows/Preferences Menu of Rodin
\begin{figure}
\centering
\includegraphics[scale=0.5]{preferences.PNG}
\caption{Rodin preferences page} 
\label{Fig: Rodin Preferences}
\end{figure}

\begin{figure}
\centering
\includegraphics[scale=0.5]{preferences2.PNG}
\caption{Rodin SMT preferences page} 
\label{Fig: Rodin SMT Preferences}
\end{figure}

\paragraph{Solver preference}
The user must provide info on solvers and additionnal ones such like if a preprocessing is used or not. Pushing the \textit{Create} button leads to the following window:

\begin{figure}
\centering
\includegraphics[scale=0.5]{preferences3.PNG}
\caption{Rodin SMT preferences page, new solver} 
\label{Fig: New solver in SMT preferences}
\end{figure}

\paragraph{Solver preference creation}
The user must provide following info:

\begin{itemize}
\item An Id,
\item A path,
\item Arguments used to call the solver,
\item On which SMT-LIB version the solver will be used.   
\end{itemize}

\paragraph{Solver selection}
The user must then select the solver that will be used:

\begin{figure}
\centering
\includegraphics[scale=0.5]{preferences4.PNG}
\caption{Rodin SMT preferences page, solver selection} 
\label{Fig: Solver selection in SMT preferences}
\end{figure}

\paragraph{Solver launching}
To launch the SMT solver on the selected Proof Obligation, the user must press the SMT button:
\begin{figure}
\centering
\includegraphics[scale=0.5]{smt.PNG}
\caption{Rodin SMT button} 
\label{Fig: Launch SMT in Rodin}
\end{figure}

\paragraph{Solver result}
If the SMT solver suceeded, the status will be updated in the proof tree:

\begin{figure}
\centering
\includegraphics[scale=0.5]{smt2.PNG}
\caption{Rodin SMT results} 
\label{Fig: SMT results}
\end{figure}




%% ----------------------------- %%
%% Conclusion %%
%% ----------------------------- %%
\section{Conclusion}

\paragraph{Future work.} 
The next steps planned are the following:
\begin{itemize}
\item Keep a link between hypotheses in Event-B format and those in SMT format. 
Such improvement will give the possibility to memorize for example, hypotheses needed to carry through a Rodin proof obligation. If an hypothesis changed, the proof obligation will not need to be done again, if the hypothesis is not among the needed ones. To achieve that, we will certainly have to use label on hypotheses introduced in SMT-Lib v2.0.
\item Get back counterexamples from SMT solvers (supporting SMT-LIB v2.0) to use them in 
Rodin.
\item Recommend an SMT solver to user in accordance to Proof Obligation type. Performances 
between solvers differ from one to another depending on the considered logic. It would be interesting to 
guide the user on a specific solver after a proof obligation analysis.  
\item Introduce an additional step before the final translation from FOL to SMT-LIB, 
which would be a simplification operation on FOL formulas. More precisely, there 
is no doubt that the $(0 >= 0) \wedge (0 <= x)$ formula can be easily simplified as 
$0 <= x$. Such simplifications are not considered in this document, and are only 
introduced here as topic of interest for further revisions.
\item Perform benchmarks on solvers in order to classify solvers (supported theories, SMT-lib supported version, performance according to chosen theories,...)

\end{itemize}

\paragraph{Acknowledgements.} 
The author thanks David D\'eharbe and V\'itor Alc\^antara de Almeida (Universidade Federal do Rio Grande do Norte, Natal, RN, Brazil), whose work on integrating SMT solvers in Rodin \cite{RODINSMT10} has provided a solid basis and has largely contributed to the contents of this deliverable.

%% ----------------------------- %%
%% Index and references
%% ----------------------------- %%
\nocite{*}
\bibliographystyle{plain}
\bibliography{biblio}

%% ----------------------------- %%
%% Index and references
%% ----------------------------- %%
\appendix
\makeatletter
\def\@seccntformat#1{Appendix~\csname the#1\endcsname:\quad}
\makeatother
\section{Benchmarking Procedure and Results}
\label{Bench}

     
\end{document}
